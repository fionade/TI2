%%!TEX TS-program = latex

\documentclass[11pt]{article}
\usepackage[a4paper]{geometry}
%\usepackage[pdftex]{graphicx}
\usepackage[utf8]{inputenc}

\usepackage{amssymb}
\usepackage{amsmath}
\usepackage{amsthm}

\usepackage{caption}
\usepackage{subcaption}

\usepackage{array}

% Theorem Styles
\theoremstyle{definition}
\newtheorem{theorem}{Theorem}[section]
\newtheorem{lemma}[theorem]{Lemma}
%\newtheorem{proposition}[theorem]{Proposition}
%\newtheorem{corollary}[theorem]{Corollary}
%% Definition Styles
\theoremstyle{definition}
\newtheorem{definition}{Definition}[section]
%\newtheorem{example}{Example}[section]
%\theoremstyle{remark}
%\newtheorem{remark}{Remark}

\usepackage{longtable}

% Graphen etc.
\usepackage{pst-node}
\usepackage{pst-tree}

\newcommand{\rcirclenode}[3]{
	\circlenode{\# 1}{\parbox{\# 2}{\centering\# 3}}
}

\usepackage{hyperref}

% Das Dokument beginnt hier
\begin{document}

\setlength{\parindent}{0pt}

% Erstelle eine Hauptüberschrift

\title{Theoretische Informatik 2 Exercises}
\author{}
\date {}

\maketitle

\section*{Exercise 32}

\paragraph{Given:}
$ f: \{0, 1\}^\ast \rightarrow \{0, 1\}^\ast $ one-way permutation

\paragraph{Task:} Show that $ f^k $ is one-way $ \forall k \in \mathbb{N} $

\begin{proof}

by induction on $ k $.

\begin{itemize}
\item $ k = 1 $
\item $ k > 1 $

	$ k = 2 $
	
	By hypothesis, we have that for every probabilistic polynomial time algorithm $ A $, the following holds $ \forall n \in \mathbb{N} $:
	
	\[ \mathbb{P}(f (A(f(x))) = f(x) ) < \varepsilon(n),~ x \in \{0, 1\}^n \]
	
	Consider an arbitrary probabilistic polynomial algorithm $ B $. Observe that 
	\[ \mathbb{P}(f^2 (B (f^2(x)) = f^2(x) ) = P(f(f(B(f(f(x))))) = f(f(x))) \]
	
	A permutation is bijective, so there exists an inverse function $ f^{-1} $ $ \rightarrow $ apply $ f^{-1} $ on both sides and yield
	
	\[ \mathbb{P} (f(B(f(f(x)))) = f(x)) = \mathbb{P}[ f(\underbrace{B \circ f(x))}_{\text{a prob. polyn. alg.}} = f(x)] < \varepsilon \]
	$ \Rightarrow $ defining $A:= B \circ f $ we can show the assumption
\end{itemize}

\end{proof}

\section*{Exercise 34}

\begin{enumerate}
\item[a.] Prove that PCP$(0, \log n) = $ P
	% no random bits allowed
		
	\begin{itemize}
	\item ``P $ \subseteq $ PCP$(0, \log n) $'' \newline
	
		Let $ L \in $ P. A verifier $ V $ of PCP$(0, 0) \subseteq $ PCP$(0, \log n) $ has polynomial running time and can decide $ L $.
	
	\item ``P $ \supseteq $ PCP$(0, \log n) $'' \newline
	
	Algorithm:
	
		For each proof ($ O(2^{\log n}) = O(n)) $
			If $ V $ accepts, accept
		Else
			Reject
			
	Total running time: $ O(n) \cdot \text{poly}(n) = \text{poly}(n) $
	\end{itemize}

\item[b.] Prove that PCP$(0, \text{poly}( n)) = $ P
	\begin{itemize}
	\item There is a verifier $ V $ polynomial, deterministic
	\item $ V $ decides $ L $
	\item $ P(...) < \frac{1}{2} $ means 0 (no random bits)
	\end{itemize}

\end{enumerate}


\section*{Exercise 35}

Show that PCP$(\log n, 1) \subseteq $ NP.

\begin{proof}

Let $ L \in $ PCP$(log n, 1) $. We build a non-deterministic TM $ M $ which works as follows:
\begin{enumerate}
\item $ M $ generates non-deterministically all the proofs of length at most $ ^\cdot 2^{O(\log n)} $. This can be done in $ O(\log n) $ steps.
\item $ M $ generates non-deterministically all the $ 2^{O(\log n)} $ possible sequences of coin tosses
\item $ M $ emulates the verifier on all these toss sequences ($ M \in $ PCP)
\item $ M $ accepts $ \Leftrightarrow $ the verifier accepts on all these sequences
\end{enumerate}

$ \rightarrow M $ runs in $ 2^{O(\log n )} = \bigcup \limits_{c \geq 0} n^c \Rightarrow L \in $ NP

\end{proof}

\section*{Exercise 37}

\paragraph{Task:} Provide a PCP(poly$(n,1)$ verifier for the complement of the graph isomorphism problem.

$ \overline{\text{GI}} $ is the complement of GI, i.e. the language consisting of non-isomorphic graphs. \newline

\paragraph{Input:} graphs $ G_0, G_1 $ which both have $ n $vertices and $ m $ edges.

The verifier expects the proof $ \Pi $ to contain a bit $ \Pi(H) $ $(\in \{0, 1\}) $ for each labeled graph with $ n $ nodes such that $ \Pi(H) \in \{0, 1\} $ corresponding to whether $ H \cong G_0 $ or $ H \cong G_1 $ \newline

$ \rightarrow $ in other words, $ \Pi $ can be seen as an exponentially long array of bits indexed by all possible graphs on $ n $ vertices.

Verifier picks random bit $ b \in \{0, 1\} $ and a random permutation $ \rho \in S_n $ \newline

Apply $ \rho $ to vertices of $ G_b $.

$ \rightarrow $ Leads to graph $ H \cong G_b $

Verifier queries the proof bit $ \Pi(H) $ and accepts if this bit equals $ b $

\begin{enumerate}
\item[Case 1:] $ G_0 \not \cong G_1 $ \newline
	In this case $ \Pi $ can be set up such that the verifier accepts with probability 1
\item[Case 2:] $ G_0 \cong G_1 $ \newline
	The probability that any proof makes the verifier accept is at most $ \frac{1}{2} $
\end{enumerate}

\section*{Exercise 39}

\[ (x_1 \vee x_2 \vee x_3) \wedge (\overline{x_1} \vee \overline{x_3} \vee x_4) \wedge (x_2 \vee x_3 \vee \overline{x_4}) \]

\begin{enumerate}
\item[a.] \[ q = (1 - x_1) \cdot x_2 \cdot (1 - x_3) + x_1 \cdot x_3 \cdot (1 - x_4) + (1 - x_2) \cdot (1 - x_2) \cdot x_4 \]
	\[ = x_2 - x_1 \cdot x_2 - x_2 \cdot x_3 + x_1  \cdot x_2 \cdot x_3 + x_1 \cdot x_3 - x_1 \cdot x_3 \cdot x_4 + x_4 - x_2 \cdot x_4 - x_3 \cdot x_4 + x_2 \cdot x_3 \cdot x_4\]
	\[ = x_2 + x_4 - x_1 \cdot x_2 + x_1 x_3 - x_2 x_3 - x_2 x_4 + x_1 x_2 x_3 - x_1 x_3 x_4 + x_2 x_3 x_4 \]
\item[b.] \[ I_q^1 = \{2, 4\} \]
 \[ I_q^2 = \{(1,2), (1, 3), (2, 3), (2, 4), (3, 4)\} \]
  \[ I_q^3 = \{(1, 2, 3), (1, 3, 4), (2, 3, 4)\} \]
\item[c.] \[ a = (1, 0, 1, 1) \]
	\[ \gamma + L_1^q(a_1^1) + L_2^q(c_1^2) + L_2^(c_q^3) \]
	
	\[ C_{q_i}^1 = \left\{ \begin{array}{c l}1 & i \in I_1^1 \\ 0 & i \not \in I_1^1 \end{array} \right. \]
	
	\[ \gamma_q = 0 \]
	(since the polynomial has no constant term)
	
	\[ L_1^a (c_q^1) = a_2 + a_4 = 1 \]
	\[ L_2^a (c_q^1) = a_1 a_2 + a_1 a_3 + a_2 a_3 + a_2 a_4 + a_3 a_4 = 0 (2) \]
	\[ L_3^a (c_q^1) = a1 a_2 a_3 + a_1 a_3 a_4 + a_2 a_3 a_4 = 1 \]
	
	\[ \sum = 0 + 1 + 0 + 1 = 0. (2) \]
	
	Insert into the original formula
	
	0 + 0 + 0 = 0

\end{enumerate}

\end{document}










